\myChapter{Applied methods... title TBC}\label{chap:softc} 
\chaptermark{Applied methods... title TBC} 

\begin{flushright}{\slshape
    ...} \\ \medskip
    --- {}
\end{flushright}

\minitoc\mtcskip
\vfill

\lettrine{S}{ince} the final outcome from the application of the methodology that is proposed in this thesis is a set of rules, that will aid the CSOs in the process of reviewing and extending the set of security policies, we present the most suitable techniques to produce this outcome, taking into account the input data. The input data are be the actions that the employees make with their devices in certain situations, hence the final problem is that of classification, to see if they should be allowed or not.

Therefore, in this chapter we will look over the techniques that take part in the process, from selecting the data in the database to actually obtaining the set of rules and visualise them, via data preprocessing and transformation. Furthermore, the soft computing techniques that have been applied in research for the problem of classification -- extraction of rules -- and visualisation are also detailed.

It is important to note that this process, known as ``Knowledge Discovery in Databases'' (KDD) is not the proposed methodology itself, but part of it. These are the necessary steps to obtain the rules, but the proposed algorithms can be used or not depending on the use case, which will be defined later.

\section{The Knowledge Discovery in Databases process}

\cite{fayyad1996data}

\section{Data selection}

\section{Data preprocessing}

\section{Data transformation}

\section{Soft computing techniques applied to data mining and visualisation}

\subsection{Classification in the data mining process}

\subsection{Data visualisation and interpretation}