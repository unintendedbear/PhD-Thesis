\myChapter{Introduction}\label{chap:introduction}
\begin{flushright}{\slshape
    .} \\ \medskip
    --- {}
\end{flushright}
\minitoc\mtcskip
\vfill



\section{Goals of this Thesis} %FERGU: esto es un párrafo que describe el objetivo de la tesis.

\lettrine{T}{he} goal of this thesis is to develop a methodology of data preprocessing, behaviour analysis, and consequent rule creation in a Bring Your Own Device (BYOD) context. This methodology proposes using a combination of Machine Learning (ML) and Soft Computing (SC) techniques over context information related to a user interacting with a device, making benefit of the feature of rule model creation in some of these techniques. Finally, this methodology will be used to help evolving an initial, simple, and fixed set of security rules into a new set of rules that will aid in the discovery of dangerous situations.

\section{Motivation}
\label{sec:intro:byod}

\lettrine{T}{he} fast pace of introduction of new technologies, including the so-called \textit{smart devices}, has led modern computing to move in a short period of time to smaller, more reliable, and faster devices such as smartphones, laptops, and tablets. At the same time, the use people give them is more varied and, among others, has given birth to the form of use known as the ``Bring Your Own Device'' (BYOD) concept or philosophy. Since its first appearance in research \cite{ballagas2004byod} as a way to name the interaction between people's devices and a public display, such as a piece of art or advertising, it has become a very popular practise, integrated into companies \cite{thomson2012byod} and even schools \cite{song2014bring}.

More precisely, in the corporate world the BYOD practice refers to allowing employees to use their personal laptops, smartphones, tablets, and other mobile devices for work-related tasks, while not necessarily being in the workplace. This has many advantages \cite{singh2012byod}; among them it is worth to mention saving costs -- since the company saves money on high-priced devices that it would normally be required to purchase for their employees --, and increasing flexibility and worker productivity as employees will not be asked to haul around while on the road multiple devices to satisfy both their personal and work needs. Other advantages are tied to the increase of worker satisfaction, attracting the best candidates, and the increase of engagement in the workplace and after hours \cite{singh2012byod}.

It has, however, a big disadvantage concerning security, since potentially unsecured devices from unaware users might interact with important company assets. Clearly the uncontrolled access to internal networks by the personal devices, for which companies have control limits due to privacy concerns \cite{miller2012byod}, exposes the companies to security risks such as data leakage, improper decommissioning, phishing that is focused on a specifig group or company\footnote{Also known as \textit{spear phishing}.}, surveillance, and many others possible threats\cite{lennon2012changing}.

The main manufacturers of the mentioned devices have taken advantage of this BYOD trend and have released a number of commercial products to help companies adopting the philosophy. However, they all work based on a fixed, pre-defined set of security rules which they simply apply to certain user actions.

Therefore, the main issue in the adaptation to a BYOD scenario is to obtain a high level of security, while maintaining user privacy \cite{miller2012byod}. A combination of Soft Computing and Machine Learning techniques is proposed in this thesis as a solution able to propose new security rules, that is platform independent and open source, to help companies increasing security by incorporating the new rules to the existing set and discover new relations between the events that might cause a dangerous situation.

\section{Challenges in BYOD}
\label{sec:intro:challenges}

\lettrine{E}{ven} though the term BYOD is relatively recent, as it has been mostly in the corporate world, researchers have studied its strong and weak points. Among the parts that worth improvement, the most important for this thesis are the following:

\subsection{Context awareness}
\label{subsec:context}

% Problem:
Previous papers have commented on the issue of using quality data to obtain accurate and valuable extracted knowledge, which is specially difficult when dealing with datasets taken from real world problems, where the sensors cannot work properly at a certain time. To avoid this, researchers have advanced the state of the art in the precision of the devices. For instance, in \cite{rios2015mobile}, the authors focus on the importance of having an accurate measure of the location of the devices. Although their approach seems to encroach upon employees' privacy, they implemented a mobile information system for BYOD adaptation, and tested it in both Android and iOS devices. The authors concluded that the efficiency of their system, along with the possibilities of the hardware in the devices, results in a location
error of less than 50 meters with a 95\% confidence level. These promising results allow the companies to use the location of their employees to apply certain security policies.

% How can we benefit and how can we help
In our case, an accurate detection of the location could be used to obtain more information about the happening event, and that could increase the classification accuracy. At the same time, the extracted rules as a result from the methodology we propose in this thesis can help the company in indentifying certain locations as ``dangerous'' or ``safe''.

\subsection{User behaviour}
\label{subsec:behav}

% Problem:
Even not on purpose, the users of the system are considered the main hazard due to their ignorance about company security rules  \cite{Adams_users99}. They want to benefit from the BYOD philosophy, so that they can balance their private and work life, but they usually are not familiar with the Corporate Security Policy (CSP) of their company.  Actually, the employees have a natural tendency to comply with the security policies \cite{Sip_SecPriv07,Bulg_SecPol10,AlOmari_SecPol12}, and that good intention increases by educating or training them in information security awareness  \cite{Shaw_SecAware09}, and decreases applying too much sanctions when a misuse or abuse occurs \cite{Her_SecPol09}.

% How can we benefit and how can we help
Soft Computing techniques, such as data visualisation, can help in understanding the behaviour of the users and the risks it might carry. We propose the use of these techniques, among others to aid the CSO presenting conclusions to the company, also helping them to identify possible threats.

\subsection{User privacy}
\label{subsec:user_priv}

% Problem:
When monitoring a system for security purposes, the deeper the analysis is, the best chances of finding the threat we have. A BYOD scenario implies monitoring users own devices, having a direct impact in their privacy \cite{Miller12Privacy}. And what is more, the users are very concerned about their privacy, and when asked about installing ``monitoring software'', they are very mindful about it \cite{Miller12Privacy, ali2015analysis}. Most of the state of the art and commercial solutions for BYOD address this problem in different ways \cite{de2015corporate}, but are not transparent to the user.

% How can we benefit and how can we help
The need for developing transparent, open software is then crucial, so that the users have the opportunnity to know which data is being monitored and what use is given over that data. In addition, in this thesis we have proposed the extraction of new attributes from the initial, monitored ones, to keep the information they can give but maintaining the privacy.

\subsection{Platform independence}
\label{subsec:platf_ind}

% Different OSs, architectures...

\subsection{Open science}
\label{subsec:openS}

% I know i'm barely copying Pablo but here we can talk about not available datasets and their problems (privacy).

\subsection{Applications}
\label{subsec:apps}

\section{Objectives}                     
\label{sec:intro:objs}

...

\newcommand{\objectivescenarios}{To describe the target environments, or use cases, to take into account in order to develop a the metrics, techniques, and problems to solve.} % to describe or to define?

 \subsection*{Objective 1: \objectivescenarios}
\label{subsec:intro:obj:problems}

\newcommand{\objectivemetrics}{To define a set of metrics able to measure the efficiency of a particular set of security rules.}

\subsection*{Objective 2: \objectivemetrics} 
\label{subsec:intro:obj:methodology}

\newcommand{\objectivetechniques}{To choose which Soft Computing and Machine Learning techniques suit best for every defined scenario.}

\subsection*{Objective 3: \objectivetechniques}
\label{subsec:intro:obj:fwork}

\newcommand{\objectiveresearch}{} 

\subsection*{Objective 4: \objectiveresearch}
\label{subsec:intro:obj:applications}



\section{Structure of the thesis}
\label{sec:intro:structure}


%\begin{SCfigure}[tb]
%\centering
% \includegraphics[scale =0.3] {gfx/intro/tesispiramide.pdf}
%\caption{Summary of the objectives of this thesis.}
%\label{fig:intro:piramid}
%\end{SCfigure}
